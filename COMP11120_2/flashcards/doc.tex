\documentclass[frontgrid]{flacards}
\usepackage{color}
\usepackage{amsmath,amssymb}
\usepackage{tabularx}

\definecolor{light-gray}{gray}{0.75}

\newcommand{\frontcard}[1]{\textcolor{light-gray}{\colorbox{light-gray}{$#1$}}}
\newcommand{\backcard}[1]{#1} 

\newcommand{\flashcard}[1]{% create new command for cards with blanks
  \card{% call the original \card command with twice the same argument (#1)
    \let\blank\frontcard% but let \blank behave like \frontcard the first time
    #1
  }{%
    \let\blank\backcard% and like \backcard the second time
    #1
  }%
}

\begin{document}

\pagesetup{2}{4}

%===============================================================================
%===============================================================================
%==================================Induction====================================
%===============================================================================
%===============================================================================

\card{
	Give the identity for:
	\[
		\sum^n_{i=0} i
	\]
}{
	\[
		\frac{n(n+1)}{2}
	\]
}

\card{
	Give the identity for:
	\[
		\sum^n_{i=0} x^i
	\]
}{
	\[
		\frac{x^{n+1} - 1}{x-1}
	\]
}

\card{
	Give the identity for:
	\[
		\sum^n_{i=0} i(i+1)
	\]
}{
	\[
		\frac{n(n+1)(n+2)}{3}
	\]
}

\card{
	Give the identity for:
	\[
		\sum^n_{i=0} i^2
	\]
}{
	\[
		\frac{n(n+1)(2n+1)}{6}
	\]
}

%===============================================================================
%===============================================================================
%===================================Matrices====================================
%===============================================================================
%===============================================================================

%===============================================================================
% Sets of axes
%===============================================================================

\flashcard{
  In a right handed axes, your \blank{thumb} should point in the direction of
  the \blank{x} axis, your \blank{fore finger} should point in the direction of 
  the \blank{y} axis, and your \blank{middle finger} should point in the
  direction of the \blank{z} axis.
}

%===============================================================================
% Matrices basics
%===============================================================================

\card{
  How do you add two matrices together? What are the conditions for matrix
  addition?
}{
  In order for you to be able to add two matrices together, they must both
  have the same dimensions. Then, just add value of each position in one
  matrix to the corresponding position in the other matrix.

  \[
    (A + B)_{ij} = A_{ij} + B_{ij}
  \]
}

\card{
  How can a matrix be scaled?
}{
  Just multiply each cell in the matrix by the scaling factor.
}

\card{
  How is matrix subtraction performed? What are the conditions?
}{
  In order to subtract one matrix from another, just scale the first one by a
  factor of $-1$ and add them together. As with addition, the matrices must
  have the same dimensions.

  \[
    (A - B)_{ij} = (A + (-1)B)_{ij}
  \]
}

\card{
  What is the condition for matrix multiplication?
}{
  The number of columns in the first matrix must equal the number of rows in
  the second.
}

\card{
  How do we determine the value of the cell $i,j$ when multiplying a matrix
  $A$ by another matrix $B$?
}{
  It depends on the values in the $i$'th row of $A$ and the $j$'th column of
  $B$.
  \[
    C_{ij} = \sum^{n}_{k=1} A_{ik}B_{kj}
  \]
}

\card{
  What is the identity matrix?
}{
  
  A square matrix where every cell is set to zero, except from those on the
  diagonal from the top left to the bottom right, where they are set to one.
  For example:
  \[
    \left(
      \begin{array}{ccc}
        1 & 0 & 0\\
        0 & 1 & 0\\
        0 & 0 & 1\\
      \end{array}
    \right)
  \]
}

%\card{
% How do you find the inverse of a matrix?
%}{
% In order to find the inverse of a matrix, you scale it be $-1$ and then
% multiply it by itself:
% \[
%   A^{-1} = A(-1)A
% \]
%}

\card{
  How do you work out the transpose of a matrix? What is the notation?
}{
  To work out the transpose of a matrix, you simply rotate everything
  clockwise by $90^{\circ}$. E.g.:
  \[
    \begin{split}
    A &= \left(
        \begin{array}{cc}
          1 & 3\\
          0 & 5\\
          8 & 7\\
        \end{array}
      \right)\\
    A^T &= \left(
        \begin{array}{ccc}
          1 & 0 & 8\\
          3 & 5 & 7\\
        \end{array}
      \right)
    \end{split}
  \]
}

\card{
  What is a symmetric matrix?
}{
  A matrix $M$ is symmetric when $M = M^T$. This means it is symmetric along
  the main diagonal.\\\vspace{1em}
  Alternately, you could say that $M_{ij} = M_{ji}$.
}

\card{
  Define the zero matrix.
}{
  A matrix where all of the cells have a value of zero.

  \[
    \left(
      \begin{array}{ccc}
        0 & 0 & 0\\
        0 & 0 & 0\\
        0 & 0 & 0\\
      \end{array}
    \right)
  \]
}

\card{
  Define the term {\bf commuting matrices}.
}{
  A pair of matrices $A,B$ where $AB = BA$
}

\card{
  Is matrix multiplication associative?
}{
  Yes.
  \[
    A(BC) = (AB)C
  \]
}

\card{
  Is matrix multiplication a distributive operation?
}{
  Yes.
  \[
    \begin{split}
      A(B + C) = AB + AC\\
      (B + C)A = BA + CA
    \end{split}
  \]
}

\card{
  Does the following hold between the two matrices $A$ and $B$:
  \[
    (AB^T) = A^TB^T
  \]
}{
  Yes.
}

\card{
  Does the following matrix represent a point or a vector?
  \[
    \left[
      \begin{array}{c}
        4\\
        2\\
        3\\
        1
      \end{array}
    \right]
  \]
}{
  The point $(4,2,3)$.
}

\card{
  Does the following matrix represent a point or a vector?
  \[
    \left[
      \begin{array}{c}
        4\\
        2\\
        3\\
        0
      \end{array}
    \right]
  \]
}{
  The vector $4i + 2j + 3k$.
}

%===============================================================================
% Affine transformations and matrices
%===============================================================================

\card{
  What cells are the same for all affine transformation matrices?
}{
  The bottom three/four.
  \[
    \left(
      \begin{array}{cccc}
        a_{11} & a_{12} & a_{13} & b{1}\\
        a_{21} & a_{22} & a_{23} & b{2}\\
        a_{31} & a_{32} & a_{33} & b{3}\\
        0      & 0      & 0      & 1
      \end{array}
    \right)
    ~and~
    \left(
      \begin{array}{ccc}
        a_{11} & a_{12} & b{1}\\
        a_{21} & a_{22} & b{2}\\
        0      & 0      & 1
      \end{array}
    \right)
  \]
}

\card{
  What is the matrix that will perform an affine translation?
}{
  \[
    \left(
      \begin{array}{cccc}
        1 & 0 & 0 & x\\
        0 & 1 & 0 & y\\
        0 & 0 & 1 & z\\
        0 & 0 & 0 & 1\\
      \end{array}
    \right)
  \]
}

\card{
  What is the matrix that will perform an affine scaling?
}{
  \[
    \left(
      \begin{array}{cccc}
        x & 0 & 0 & 0\\
        0 & y & 0 & 0\\
        0 & 0 & z & 0\\
        0 & 0 & 0 & 1\\
      \end{array}
    \right)
  \]
}

\card{
  How is it possible to combine two or more affine transformation matrices?
}{
  Multiply them together in the reverse order for which they are to be applied.
}

\card{
  How do you do transformation matrix powers?
}{
  You apply the matrix $n$ times, where $n$ is the power.
}

\card{
  What is the identity transformation?
}{
  A transformation that does nothing.
}

\card{
  What do you get if you multiply a matrix and the inverse of the same matrix
  together?
}{
  The identity matrix.
}

\card{
  What is the general formula for a matrix to scale by $\beta_{xyz}$ times about
  a point $(p,q,r)$?
}{
  \[
    \left(
      \begin{array}{cccc}
        \beta_x & 0       & 0       & p(1-\beta_x)\\
        0       & \beta_y & 0       & q(1-\beta_y)\\
        0       & 0       & \beta_z & r(1-\beta_z)\\
        0       & 0       &  0      & 1\\
      \end{array}
    \right)
  \]
}

\card{
  How do you rotate about a point in 2D space?
}{
  \begin{tabular}{l c}
    1 & Translate the point to the origin\\
    2 & Rotate through $\theta$ as appropriate\\
    3 & Translate the origin back to the point to rotate around\\
  \end{tabular}
}

\card{
  What is a matrix to rotate about the origin in 2D space?
}{
  \[
    \left(
      \begin{array}{ccc}
        \cos\theta & -\sin\theta & 0\\
        \sin\theta & \cos\theta  & 0\\
        0          & 0           & 1\\
      \end{array}
    \right)
  \]
}

\card{
  What is the formula for a 3D rotation about the $x$ axis?
}{
  \[
    \left(
      \begin{array}{cccc}
        1 & 0          & 0           & 0\\
        0 & \cos\theta & -\sin\theta & 0\\
        0 & \sin\theta & \cos\theta  & 0\\
        0 & 0          & 0           & 1
      \end{array}
    \right)
  \]
}

\card{
  What is the formula for a 3D rotation about the $y$ axis?
}{
  \[
    \left(
      \begin{array}{cccc}
        \cos\theta  & 0 & \sin\theta  & 0\\
        0           & 1 & 0           & 0\\
        -\sin\theta & 0 & \cos\theta  & 0\\
        0           & 0 & 0           & 1
      \end{array}
    \right)
  \]
}

\card{
  What is the formula for a 3D rotation about the $z$ axis?
}{
  \[
    \left(
      \begin{array}{cccc}
        \cos\theta  & -\sin\theta & 0 & 0\\
        \sin\theta  & \cos\theta  & 0 & 0\\
        0           & 0           & 1 & 0\\
        0           & 0           & 0 & 1
      \end{array}
    \right)
  \]
}

\card{
  How do you undo rotations?
}{
  Just do the same rotation as before except rotate through $-\theta$ instead of
  $\theta$. Make sure you're rotating about the same point or axis.
}

\card{
  What is the general method for rotating around a line $L$ in three dimensions?
}{
  \begin{tabularx}{0.5\textwidth}{l X}
    1 & Translate $L$ so that it goes through the origin.\\
    2 & Rotate around the $x$ axis so that the line lies in the x-y plane and
        $z=0$\\
    3 & Rotate around the $z$ axis to make the line lie on the $x$ axis.\\
    4 & Perform the rotation of $\theta^{\circ}$ on the $x$ axis.\\
    5 & Reverse step 3\\
    6 & Reverse step 2\\
    7 & Reverse step 1\\
  \end{tabularx}
}

\card{
  What is the matrix to reflect points along a line $ax + by = f$?
}{
  \[
    \left[
      \begin{array}{ccc}
        1 - \frac{2a^2}{a^2 + b^2} & -\frac{2ab}{a^2 + b^2}     & \frac{2af}{a^2 + b^2}\\
        -\frac{2ab}{a^2 + b^2}     & 1 - \frac{2b^2}{a^2 + b^2} & \frac{2bf}{a^2 + b^2}\\
        0                          & 0                          & 1\\
      \end{array}
    \right]
  \]
}

\card{
  How is a reflection undone?
}{
  Just re-apply the reflection. An interesting property of any reflection matrix
  $R$, is that $R^2 = I$, where $I$ is the identity matrix.
}

\card{
  What is the formula for a 3D reflection along the line $ax + by + cz = f$?
}{
  \[
    \left[
      \begin{array}{cccc}
        1 - 2\frac{a^2}{a^2 + b^2 + c^2} & - 2\frac{ab}{a^2 + b^2 + c^2}    & - 2\frac{ac}{a^2 + b^2 + c^2}    & 2\frac{af}{a^2 + b^2 + c^2}\\
        - 2\frac{ab}{a^2 + b^2 + c^2}    & 1 - 2\frac{b^2}{a^2 + b^2 + c^2} & - 2\frac{bc}{a^2 + b^2 + c^2}    & 2\frac{bf}{a^2 + b^2 + c^2}\\
        - 2\frac{ac}{a^2 + b^2 + c^2}    & - 2\frac{bc}{a^2 + b^2 + c^2}    & 1 - 2\frac{c^2}{a^2 + b^2 + c^2} & 2\frac{cf}{a^2 + b^2 + c^2}\\
        0 & 0 & 0 & 1\\
      \end{array}
    \right]
  \]
}

\card{
  What is the formula for projecting a 3D point onto the plane
  $ax + by + cz = f$?
}{
  \[
    \left[
      \begin{array}{cccc}
        1 - \frac{a^2}{a^2 + b^2 + c^2} & - \frac{ab}{a^2 + b^2 + c^2}    & - \frac{ac}{a^2 + b^2 + c^2}    & \frac{af}{a^2 + b^2 + c^2}\\
        - \frac{ab}{a^2 + b^2 + c^2}    & 1 - \frac{b^2}{a^2 + b^2 + c^2} & - \frac{bc}{a^2 + b^2 + c^2}    & \frac{bf}{a^2 + b^2 + c^2}\\
        - \frac{ac}{a^2 + b^2 + c^2}    & - \frac{bc}{a^2 + b^2 + c^2}    & 1 - \frac{c^2}{a^2 + b^2 + c^2} & \frac{cf}{a^2 + b^2 + c^2}\\
        0 & 0 & 0 & 1\\
      \end{array}
    \right]
  \]
}

\card{
  Why is it impossible to reverse a projection?
}{
  Projection matrices are examples of singular matrices, and therefore don't
  have an inverse. If you think about it, multiple points could be mapped onto
  the same point in the plane anyway, so undoing a projection wouldn't make
  sense.
}

\flashcard{
  The determinant is defined only for \blank{square} matrices.
}

\card{
  What is the formula to find the determinant of a matrix $M$ where $n>2$?
}{
  \[
    det~M = \sum^n_{j=1}(-1)^{j+1} \cdot M_{1j} \cdot det~\overline{M}_{1j}
  \]
}

\card{
  How do you find $\overline{M_{ij}}$?
}{
  If $M$ is an $n \times n$ matrix, then $M_{ij}$ is the $(n - 1) \times (n -
  1)$ matrix obtained from $M$ by throwing away the $i$th row and $j$th column.
}

\card{
  How is the determinant defined for a $2 \times 2$ matrix such as:\\
  \[
    \left(
      \begin{array}{cc}
        a & b\\
        c & d\\
      \end{array}
    \right)
  \]
}{
  \[
    ad - bc
  \]
}

\card{
  How can you find the determinant of a matrix in row echerlon form?
}{
  Multiply the cells along the main diagonal together.
}

\card{
  In order to get a matrix into row echerlon form so that you can find the
  determinant, you can do what two operations? What are their effects?
}{
  \begin{tabularx}{0.4\textwidth}{l X}
    - & You can swap rows, but doing so negates the determinant.\\
    - & You can multiply one row by another, this does not affect the
        determinant.\\
  \end{tabularx}
}

\card{
  What does the determinant of a matrix signify in a geometric sense?
}{
  The area scale factor for 2D transformations and the volume scale factor for
  3D ones.
}

\card{
  When is a square matrix singular?
}{
  If and only if its determinant if zero.
}

\end{document} 
