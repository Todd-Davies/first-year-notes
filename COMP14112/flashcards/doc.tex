\documentclass[frontgrid]{flacards}
\usepackage{color}
\usepackage{tabularx}

\definecolor{light-gray}{gray}{0.75}

\newcommand{\frontcard}[1]{\textcolor{light-gray}{\colorbox{light-gray}{$#1$}}}
\newcommand{\backcard}[1]{#1} 

\newcommand{\flashcard}[1]{% create new command for cards with blanks
    \card{% call the original \card command with twice the same argument (#1)
        \let\blank\frontcard% but let \blank behave like \frontcard the first time
        #1
    }{%
        \let\blank\backcard% and like \backcard the second time
        #1
    }%
}

\begin{document}

\pagesetup{2}{4} 

%===============================================================================
% Robot localisation - probability revision
%===============================================================================

\card{
	If a collection of events are {\bf mutually exclusive}, then the probability of the conjunction of those events is...
}{
	$0$
}

\card{
	If a collection of events are {\bf jointly exhaustive}, then the probability of the disjunction of those events is...
}{
	$1$
}

\card{
	When does a collection of events form a {\bf partition}?
}{
	When the events are both mutually exclusive and jointly exhaustive.
}

%===============================================================================
% Robot localisation - robot questions
%===============================================================================

\card{
	Name some types of actuator that might be found on a robot.
}{
	\begin{tabular}{ll}
		- & Stepper motors\\
		- & DC motors\\
		- & Artificial muscles\\
		- & Hydraulic controls\\
	\end{tabular}
}

\card{
	Name some types of sensors that might be found on a robot.
}{
	\begin{tabular}{ll}
		- & Camera\\
		- & Bumpers\\
		- & Range finders (infra red, sonar, laser)\\
		- & Light detectors\\
	\end{tabular}
}

\card{
	Name two things that may cause the robot to incorrectly percieve its location.
}{
	\begin{tabularx}{0.5\textwidth}{lX}
		- & The sensors are noisy\\
		- & The robot may sometimes move a greater or lesser distance than it intended\\
	\end{tabularx}
}

%===============================================================================
% Robot localisation - the problem
%===============================================================================

\card{
	What is a pose?
}{
	A collection of three integers, representing the $x$ position, the $y$ position and the angle of rotation of the robot.
}

\card{
	Describe baysian updating.
}{
	If an agent recieves new information in the form of an Event $E$, then the agent should update its degrees of belief by conditionalising its probability distribution on $E$.
}

\card{
	If an agent recieves two events $E_1$ and $E_2$, and conditionalises on them both, does the order in which it conditionalises its probability distribution on the events mattter?
}{
	Nope.
}

\card{
	State the formula for total probability.
}{
	We can find out the probability of the event $E$ if we have the conditional probabilities of it with another set of events $E_0, E_1, \dots, E_n$ that form a partition on the sample space:

	\[
		p(E) = p(E|E_0)p(E_0) + p(E|E_1)p(E_1) + \dots + p(E|E_n)p(E_n)
	\]
}

\card{
	What is the formula for Bayes theorem?
}{
	\[
		p(E) = \frac{p(F|E)p(E)}{P(F)}
	\]
}

\card{
	What is the partition version of the formula for Bayes theorem?\\
	Where:\\
	\begin{tabular}{ll}
	- & $E$ is an event.\\
	- & $E_0 \dots E_n$ are events that form a partition over $\Omega$.
	\end{tabular}
}{
	\[
		p(E) = \frac{p(E|E_i)p(E_i)}{p(E|E_0)p(E_0) + \dots + p(E|E_n)p(E_n)}
	\]
}

\card{
	What's the difference between a cumulative distribution function and a probability mass function?
}{
	The cdf is cumulative, and shows the total probability of the outcomes up to and including the outcome it takes. The pmf shows only the probability of that outcome.
}

\card{
	What is the definition of conditional probability?
}{
	\[
		p(F|E) = \frac{p(F \wedge E)}{p(E)}
	\]
}

\end{document} 
