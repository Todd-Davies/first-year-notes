\documentclass{article}

\usepackage{fancyhdr}
\usepackage[parfill]{parskip}

\pagestyle{fancyplain}

\author{Todd Davies}
\title{Packets}
\date{\today}

\begin{document}

\rhead{Packets}
\lhead{\today}

\maketitle

Packets are small pieces of information that are sent over a network. When
computers send files between each other, they break the files down into packets
and send them individually.

When a computer breaks a file down into packets, it attaches and IP number (the
IP address of the sending computer) and a sequence number (the IP address of the
receiving computer). It also attaches a checksum to each packet so the receiving
computer can verify that it received the packet correctly.

Packets can be sent over networks using a variety of protocols, which are sets
of rules defining the roles of both the receiving and sending computers in the
transmission process. Once the packets arrive at the destination computer, they
are re-ordered into their sequential order.

Packets are split into three parts; header, data and trailer.

The header contains three things; the IP addresses of both the sending and
receiving computers and the length of the packet.

The data is the information contained in the packet.

The trailer contains the checksum of the packet.

Once the packet arrives at the destination, the checksum of the packet is
compared with a newly generated checksum of the received packet, and if they do
not match, then the destination computer will ask for the packet to be resent
since it must have been corrupted.

\end{document}
