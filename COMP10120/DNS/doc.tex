\documentclass{article}

\usepackage{fancyhdr}
\usepackage[parfill]{parskip}

\pagestyle{fancyplain}

\author{Todd Davies}
\title{What is DNS and how does it work?}
\date{\today}

\begin{document}

\rhead{What is DNS and how does it work?}
\lhead{\today}

\maketitle

In order for one computer to send data to another computer, the IP address of
the receiving computer must be known. The DNS system allows IP addresses to be
translated into more human readable (and rememberable) domain names.

\section*{Before DNS}

In the early days of the Internet, there were only a few hundred machines
online, so a single text file could contain all the host names and corresponding
IP addresses. This text file was managed by the Standford Research Institute and
was called Hosts.txt. System administrators would periodically copy that file
onto their own systems, and so the mapping of IP addresses to domain names was
kept consistent over the whole network.

However, as the Internet grew, it soon became too much work to manually update
the Hosts.txt file, since changes were being made every few minutes, and so the
DNS system was developed as an automated and scalable solution.

\section*{How does DNS work?}

DNS stands for Domain Name System, and is able to resolve domain names into IP
addresses and be dynamically updated.

Each device on the Internet is called a host (this can be anything, from a
gaming computer to an Internet connected toaster), and it has a unique IP
address. Just as the IP address is unique to the host, the domain name is also
unique. To allow for more variety, domain names aren't limited to one level, and
allow for many sub-domains.\marginpar{\raggedright A sub-domain is the part
before the main domain name. {\it blog} in blog.test.com for example.}

Top level domains are the last part of a domain name, such as \texttt{.com} or
\texttt{.gov.uk}. They are usually managed by top level domain servers.

In order to translate a domain name into an IP address, your computer first
constructs a DNS query containing the domain name and then sends it to the DNS
server listed in the IP settings of your computer.

When the DNS server receives the request, it checks its cache of previously
resolved names and if it contains the one it's looking for, it will send the IP
address back to your computer. If the server doesn't have the domain name
cached, then it'll send a request to the top level domain server, which will
find the specific DNS server for that domain and find the IP.

\end{document}
