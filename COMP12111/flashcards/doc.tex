\documentclass[frontgrid]{flacards}
\usepackage{color}

% For circuit diagrams
\usepackage{circuitikz}

\definecolor{light-gray}{gray}{0.75}

\newcommand{\frontcard}[1]{\textcolor{light-gray}{\colorbox{light-gray}{$#1$}}}
\newcommand{\backcard}[1]{#1} 

\newcommand{\flashcard}[1]{% create new command for cards with blanks
    \card{% call the original \card command with twice the same argument (#1)
        \let\blank\frontcard% but let \blank behave like \frontcard the first time
        #1
    }{%
        \let\blank\backcard% and like \backcard the second time
        #1
    }%
}

\begin{document}

\pagesetup{2}{4} 

\card{
	What is a hierarchy?
}{
	A hierarchy is a group of objects arranged in tiers of descending magnitude,
	importance or complexity.
}

\card{
	Define {\it digital}.
}{
	An entity that can reside in one of two states at any one time.
}

\card{
	Define {\it analogue}.
}{
	An entity that can reside in an infinite number of possible states.
}

\card{
	How many values can be represented by a binary number containing $n$ bits?
}{
	$2^n$
}

\card{
	Arrange {\tt AND}, {\tt OR} and {\tt NOT} in order of operator precedence.
}{
	{\tt NOT}, {\tt AND}, {\tt OR}
}

\card{
	What is the symbol for {\tt AND}? 
}{
	$\cdot$
	\\
	E.g. $A \cdot B$
}

\card{
	What is the symbol for {\tt OR}?
}{
	$+$
	\\
	E.g. $A + B$
}

\card{
	What is the symbol for {\tt NOT}
}{
	$\overline{\phantom{A}}$
	\\
	E.g. $\overline{A}$
}

\card{
	What is the symbol for {\tt XOR}
}{
	$\oplus$
	\\
	E.g. $A \oplus B$
}

\card{
	What is De Morgan's theorem commonly used for when designing digital 
	circuits?
}{
	Converting gates such as {\tt AND}, {\tt OR}, {\tt XOR} etc into {\tt NAND}
	and {\tt NOR} since they are cheap and fast.
}

\card{
	What is the symbol for an {\tt AND} gate?	
}{
	\begin{circuitikz} \draw
		(0,0) node[and port] (and) {};
	\end{circuitikz}
}

\card{
	What is the symbol for an {\tt OR} gate?	
}{
	\begin{circuitikz} \draw
		(0,0) node[or port] (or) {};
	\end{circuitikz}
}

\card{
	What is the symbol for an {\tt XOR} gate?	
}{
	\begin{circuitikz} \draw
		(0,0) node[xor port] (xor) {};
	\end{circuitikz}
}

\card{
	What is the symbol for an {\tt NOT} gate?	
}{
	\begin{circuitikz} \draw
		(0,0) node[not port] (not) {};
	\end{circuitikz}
}

\card{
	What is the symbol for an {\tt NAND} gate?	
}{
	\begin{circuitikz} \draw
		(0,0) node[nand port] (nand) {};
	\end{circuitikz}
}

\card{
	What is the symbol for an {\tt NOR} gate?	
}{
	\begin{circuitikz} \draw
		(0,0) node[nor port] (nor) {};
	\end{circuitikz}
}

\card{
	What's the symbol for a n:1 multiplexer?
}{
	% I can't find a way to draw it in LaTeX...
	\centering\fbox{
		\begin{minipage}{2in}
			\hfill\vspace{1in}
		\end{minipage}
	}
}

\flashcard{
	What is the truth table for binary addition?
	\begin{tabular}{|c c c|c c|}
		\hline
		$A$ & $B$ & $c_{in}$ & $S$ & $c_{out}$\\ \hline
		0 & 0 & 0 & \blank{0} & \blank{0}\\
		0 & 0 & 1 & \blank{1} & \blank{0}\\
		0 & 1 & 0 & \blank{1} & \blank{0}\\
		0 & 1 & 1 & \blank{0} & \blank{1}\\
		1 & 0 & 0 & \blank{1} & \blank{0}\\
		1 & 0 & 1 & \blank{0} & \blank{1}\\
		1 & 1 & 0 & \blank{0} & \blank{1}\\
		1 & 1 & 1 & \blank{1} & \blank{1}\\ \hline
	\end{tabular}
}

\card{
	How do you negate a binary number?
}{
	1. Invert the bits\\
	2. Add 1
}

\card{
	Convert binary $6$ to $-6$
}{
	1. Start with {\tt 0110}\\
	2. Invert the bits - {\tt 1001}\\
	3. Add 1 - {\tt 1010}
}


\card{
	Which bit is the signed bit when using 2's complement?
}{
	The left most bit.
}

\card{
	How do you subtract two binary numbers?
}{
	1. Invert the number you're subtracting\\
	2. Add 1 to the inverted number\\
	3. Add the number you're subtracting from with the inverted number.\\
	Basically, add the original number to the 2's complement negative of what
	you're taking away.
}

\card{
	What is the {\it sum-of-products}?
}{
	When a number of {\tt AND} gates are {\tt OR}'ed together.
}

\card{
	What is the {\it product-of-sums}?
}{
	When a number of {\tt OR} gates are {\tt AND}'ed together.
}

\card{
	What is the structure of a half adder (in terms of gates)?
}{
	\begin{circuitikz} \draw
	(0,2) node[and port] (and1) {}
	(0,0) node[and port] (and2) {}
	(2,1) node[or port] (or) {}
	(1, -2) node[and port] (and3) {}
	(and1.in 1) node[anchor=east] {$\overline{A}$}
	(and1.in 2) node[anchor=east] {B}
	(and2.in 1) node[anchor=east] {A}
	(and2.in 2) node[anchor=east] {$\overline{B}$}
	(and1.out) -- (or.in 1)
	(and2.out) -- (or.in 2)
	(or.out) node[anchor=west] {S}
	(and3.in 1) node[anchor=east] {A}
	(and3.in 2) node[anchor=east] {B}
	(and3.out) node[anchor=west] {$c_{out}$};
	\end{circuitikz}
}

\end{document} 
